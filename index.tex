% Options for packages loaded elsewhere
% Options for packages loaded elsewhere
\PassOptionsToPackage{unicode}{hyperref}
\PassOptionsToPackage{hyphens}{url}
%
\documentclass[
  letterpaper,
]{book}
\usepackage{xcolor}
\usepackage[margin=1in]{geometry}
\usepackage{amsmath,amssymb}
\setcounter{secnumdepth}{5}
\usepackage{iftex}
\ifPDFTeX
  \usepackage[T1]{fontenc}
  \usepackage[utf8]{inputenc}
  \usepackage{textcomp} % provide euro and other symbols
\else % if luatex or xetex
  \usepackage{unicode-math} % this also loads fontspec
  \defaultfontfeatures{Scale=MatchLowercase}
  \defaultfontfeatures[\rmfamily]{Ligatures=TeX,Scale=1}
\fi
\usepackage{lmodern}
\ifPDFTeX\else
  % xetex/luatex font selection
\fi
% Use upquote if available, for straight quotes in verbatim environments
\IfFileExists{upquote.sty}{\usepackage{upquote}}{}
\IfFileExists{microtype.sty}{% use microtype if available
  \usepackage[]{microtype}
  \UseMicrotypeSet[protrusion]{basicmath} % disable protrusion for tt fonts
}{}
\makeatletter
\@ifundefined{KOMAClassName}{% if non-KOMA class
  \IfFileExists{parskip.sty}{%
    \usepackage{parskip}
  }{% else
    \setlength{\parindent}{0pt}
    \setlength{\parskip}{6pt plus 2pt minus 1pt}}
}{% if KOMA class
  \KOMAoptions{parskip=half}}
\makeatother
% Make \paragraph and \subparagraph free-standing
\makeatletter
\ifx\paragraph\undefined\else
  \let\oldparagraph\paragraph
  \renewcommand{\paragraph}{
    \@ifstar
      \xxxParagraphStar
      \xxxParagraphNoStar
  }
  \newcommand{\xxxParagraphStar}[1]{\oldparagraph*{#1}\mbox{}}
  \newcommand{\xxxParagraphNoStar}[1]{\oldparagraph{#1}\mbox{}}
\fi
\ifx\subparagraph\undefined\else
  \let\oldsubparagraph\subparagraph
  \renewcommand{\subparagraph}{
    \@ifstar
      \xxxSubParagraphStar
      \xxxSubParagraphNoStar
  }
  \newcommand{\xxxSubParagraphStar}[1]{\oldsubparagraph*{#1}\mbox{}}
  \newcommand{\xxxSubParagraphNoStar}[1]{\oldsubparagraph{#1}\mbox{}}
\fi
\makeatother


\usepackage{longtable,booktabs,array}
\usepackage{calc} % for calculating minipage widths
% Correct order of tables after \paragraph or \subparagraph
\usepackage{etoolbox}
\makeatletter
\patchcmd\longtable{\par}{\if@noskipsec\mbox{}\fi\par}{}{}
\makeatother
% Allow footnotes in longtable head/foot
\IfFileExists{footnotehyper.sty}{\usepackage{footnotehyper}}{\usepackage{footnote}}
\makesavenoteenv{longtable}
\usepackage{graphicx}
\makeatletter
\newsavebox\pandoc@box
\newcommand*\pandocbounded[1]{% scales image to fit in text height/width
  \sbox\pandoc@box{#1}%
  \Gscale@div\@tempa{\textheight}{\dimexpr\ht\pandoc@box+\dp\pandoc@box\relax}%
  \Gscale@div\@tempb{\linewidth}{\wd\pandoc@box}%
  \ifdim\@tempb\p@<\@tempa\p@\let\@tempa\@tempb\fi% select the smaller of both
  \ifdim\@tempa\p@<\p@\scalebox{\@tempa}{\usebox\pandoc@box}%
  \else\usebox{\pandoc@box}%
  \fi%
}
% Set default figure placement to htbp
\def\fps@figure{htbp}
\makeatother





\setlength{\emergencystretch}{3em} % prevent overfull lines

\providecommand{\tightlist}{%
  \setlength{\itemsep}{0pt}\setlength{\parskip}{0pt}}



 


\makeatletter
\@ifpackageloaded{tcolorbox}{}{\usepackage[skins,breakable]{tcolorbox}}
\@ifpackageloaded{fontawesome5}{}{\usepackage{fontawesome5}}
\definecolor{quarto-callout-color}{HTML}{909090}
\definecolor{quarto-callout-note-color}{HTML}{0758E5}
\definecolor{quarto-callout-important-color}{HTML}{CC1914}
\definecolor{quarto-callout-warning-color}{HTML}{EB9113}
\definecolor{quarto-callout-tip-color}{HTML}{00A047}
\definecolor{quarto-callout-caution-color}{HTML}{FC5300}
\definecolor{quarto-callout-color-frame}{HTML}{acacac}
\definecolor{quarto-callout-note-color-frame}{HTML}{4582ec}
\definecolor{quarto-callout-important-color-frame}{HTML}{d9534f}
\definecolor{quarto-callout-warning-color-frame}{HTML}{f0ad4e}
\definecolor{quarto-callout-tip-color-frame}{HTML}{02b875}
\definecolor{quarto-callout-caution-color-frame}{HTML}{fd7e14}
\makeatother
\makeatletter
\@ifpackageloaded{bookmark}{}{\usepackage{bookmark}}
\makeatother
\makeatletter
\@ifpackageloaded{caption}{}{\usepackage{caption}}
\AtBeginDocument{%
\ifdefined\contentsname
  \renewcommand*\contentsname{Table of contents}
\else
  \newcommand\contentsname{Table of contents}
\fi
\ifdefined\listfigurename
  \renewcommand*\listfigurename{List of Figures}
\else
  \newcommand\listfigurename{List of Figures}
\fi
\ifdefined\listtablename
  \renewcommand*\listtablename{List of Tables}
\else
  \newcommand\listtablename{List of Tables}
\fi
\ifdefined\figurename
  \renewcommand*\figurename{Figure}
\else
  \newcommand\figurename{Figure}
\fi
\ifdefined\tablename
  \renewcommand*\tablename{Table}
\else
  \newcommand\tablename{Table}
\fi
}
\@ifpackageloaded{float}{}{\usepackage{float}}
\floatstyle{ruled}
\@ifundefined{c@chapter}{\newfloat{codelisting}{h}{lop}}{\newfloat{codelisting}{h}{lop}[chapter]}
\floatname{codelisting}{Listing}
\newcommand*\listoflistings{\listof{codelisting}{List of Listings}}
\makeatother
\makeatletter
\makeatother
\makeatletter
\@ifpackageloaded{caption}{}{\usepackage{caption}}
\@ifpackageloaded{subcaption}{}{\usepackage{subcaption}}
\makeatother
\usepackage{bookmark}
\IfFileExists{xurl.sty}{\usepackage{xurl}}{} % add URL line breaks if available
\urlstyle{same}
\hypersetup{
  pdftitle={Libro de Investigaciones},
  pdfauthor={Mauricio Romero},
  hidelinks,
  pdfcreator={LaTeX via pandoc}}


\title{Libro de Investigaciones}
\usepackage{etoolbox}
\makeatletter
\providecommand{\subtitle}[1]{% add subtitle to \maketitle
  \apptocmd{\@title}{\par {\large #1 \par}}{}{}
}
\makeatother
\subtitle{Una colección de artículos científicos y estudios de
investigación}
\author{Mauricio Romero}
\date{2025-09-01}
\begin{document}
\frontmatter
\maketitle

\renewcommand*\contentsname{Table of contents}
{
\setcounter{tocdepth}{2}
\tableofcontents
}

\mainmatter
\bookmarksetup{startatroot}

\chapter*{Inicio}\label{inicio}
\addcontentsline{toc}{chapter}{Inicio}

\markboth{Inicio}{Inicio}

Bienvenido al \textbf{Libro de Investigaciones}, una colección de
capítulos de investigación organizados en capítulos independientes.

\bookmarksetup{startatroot}

\chapter*{Contenido}\label{contenido}
\addcontentsline{toc}{chapter}{Contenido}

\markboth{Contenido}{Contenido}

\section*{Capítulo 1: La Palma Seismicity
2021}\label{capuxedtulo-1-la-palma-seismicity-2021}
\addcontentsline{toc}{section}{Capítulo 1: La Palma Seismicity 2021}

\markright{Capítulo 1: La Palma Seismicity 2021}

En septiembre de 2021, un salto significativo en la actividad sísmica en
la isla de La Palma (Islas Canarias, España) señaló el comienzo de una
crisis volcánica. Este estudio analiza los datos de terremotos
recopilados y publicados por el Instituto Geográfico Nacional (IGN),
revelando sismicidad que se origina en dos profundidades distintas.

\section*{Capítulo 2: Evaluating the Transfer of Information in Phase
Retrieval STEM
Techniques}\label{capuxedtulo-2-evaluating-the-transfer-of-information-in-phase-retrieval-stem-techniques}
\addcontentsline{toc}{section}{Capítulo 2: Evaluating the Transfer of
Information in Phase Retrieval STEM Techniques}

\markright{Capítulo 2: Evaluating the Transfer of Information in Phase
Retrieval STEM Techniques}

Este estudio evalúa métodos de recuperación de fase en microscopía
electrónica de transmisión de barrido (STEM), analizando técnicas como
imágenes del centro de masa, STEM de campo brillante corregido por
inclinación y métodos ptychográficos directos.

\section*{Arquitectura del Proyecto}\label{arquitectura-del-proyecto}
\addcontentsline{toc}{section}{Arquitectura del Proyecto}

\markright{Arquitectura del Proyecto}

Esta implementación en \textbf{Quarto} ofrece:

\begin{itemize}
\tightlist
\item
  ✅ \textbf{Estabilidad garantizada}: Servidor web confiable y
  funcional
\item
  ✅ \textbf{Navegación fluida}: Enlaces internos que funcionan
  correctamente
\item
  ✅ \textbf{Formato científico}: Soporte nativo para ecuaciones,
  referencias y figuras
\item
  ✅ \textbf{Exports múltiples}: HTML interactivo y PDF profesional
\item
  ✅ \textbf{Live preview}: Actualización automática durante edición
\item
  ✅ \textbf{Mantenimiento sencillo}: Configuración simple y robusta
\end{itemize}

\section*{Navegación}\label{navegaciuxf3n}
\addcontentsline{toc}{section}{Navegación}

\markright{Navegación}

Utiliza la navegación lateral o los enlaces de capítulos para explorar
los diferentes estudios de investigación incluidos en este libro.

\bookmarksetup{startatroot}

\chapter{1. La Palma Earthquakes: Seismic Analysis of Volcanic
Activity}\label{la-palma-earthquakes-seismic-analysis-of-volcanic-activity}

Evidence for Multi-Reservoir Magma Systems

\hfill\break

\phantomsection\label{abstract}
\bookmarksetup{startatroot}

\chapter{Abstract}\label{abstract}

This study analyzes seismic activity associated with the 2021 La Palma
volcanic eruption to investigate evidence for proposed multi-reservoir
magma systems. Using earthquake data from the Instituto Geográfico
Nacional, we examine spatial and temporal patterns of seismicity to
validate theoretical models of magma storage and transport. Our analysis
of 5,465 seismic events reveals distinct clustering at shallow (10-15
km) and deep (30-40 km) depths, providing strong evidence for both
crustal and mantle magma reservoirs feeding the Cumbre Vieja volcanic
system.

\bookmarksetup{startatroot}

\chapter{1. Introduction}\label{introduction}

La Palma, situated in the westernmost region of the Canary Islands
archipelago, represents one of Earth's most active volcanic systems.
Located approximately 100 km from the African coast, this Spanish
territory exemplifies ongoing oceanic island formation processes. The
island's geological evolution has been dominated by multiple phases of
volcanism, with the \emph{Cumbre Vieja} volcanic ridge---a north-south
oriented structure comprising the southern half of the
island---representing the most recent and currently active phase.

Understanding volcanic earthquake patterns is crucial for eruption
forecasting and hazard assessment. The 2021 eruption of Cumbre Vieja
provided an exceptional opportunity to study real-time seismic
signatures associated with magma movement through proposed
multi-reservoir systems.

\section{1.2 Historical Context and Volcanic
Setting}\label{historical-context-and-volcanic-setting}

\subsection{Eruption History}\label{eruption-history}

The historical volcanic record of La Palma spans over five centuries,
providing valuable insights into the long-term behavior of the Cumbre
Vieja system. Since European colonization in the late 1400s, eight major
eruptions have been documented, establishing patterns crucial for
probabilistic hazard assessment.

\begin{longtable}[]{@{}ll@{}}
\caption{Recent historic eruptions on La Palma}\tabularnewline
\toprule\noalign{}
Name & Year \\
\midrule\noalign{}
\endfirsthead
\toprule\noalign{}
Name & Year \\
\midrule\noalign{}
\endhead
\bottomrule\noalign{}
\endlastfoot
Current & 2021 \\
Teneguía & 1971 \\
Nambroque & 1949 \\
El Charco & 1712 \\
Volcán San Antonio & 1677 \\
Volcán San Martin & 1646 \\
Tajuya near El Paso & 1585 \\
Montaña Quemada & 1492 \\
\end{longtable}

This equates to an eruption on average every 79 years up until the 1971
event. The probability of a future eruption can be modeled by a Poisson
distribution:

\[
p(x)=\frac{e^{-\lambda} \lambda^{x}}{x !}
\]

Where \(\lambda\) is the number of eruptions per year,
\(\lambda=\frac{1}{79}\) in this case. The probability of a future
eruption in the next \(t\) years can be calculated by:

\[
p_e = 1-\mathrm{e}^{-t \lambda}
\]

So following the 1971 eruption the probability of an eruption in the
following 50 years --- the period ending this year --- was 0.469. After
the event, the number of eruptions per year moves to
\(\lambda=\frac{1}{75}\) and the probability of a further eruption
within the next 50 years (2022-2071) rises to 0.487 and in the next 100
years, this rises again to 0.736.

\subsection{Theoretical Framework: Multi-Reservoir Magma
Systems}\label{theoretical-framework-multi-reservoir-magma-systems}

Previous geophysical and petrological investigations have proposed a
conceptual model involving two primary magma storage zones beneath
Cumbre Vieja:

\begin{enumerate}
\def\labelenumi{\arabic{enumi}.}
\tightlist
\item
  \textbf{Deep mantle reservoir} (30-40 km depth): Primary magma storage
  and differentiation zone
\item
  \textbf{Shallow crustal reservoir} (10-20 km depth): Secondary storage
  feeding eruptions
\end{enumerate}

This hierarchical system suggests that magma ascends from the deep
reservoir, undergoes further processing in the shallow chamber, and
eventually reaches the surface during eruptions. Seismic monitoring
provides a unique opportunity to test this model through analysis of
earthquake depth distributions and temporal patterns.

\bookmarksetup{startatroot}

\chapter{2. Objectives and Scope}\label{objectives-and-scope}

This study aims to:

\begin{itemize}
\tightlist
\item
  Analyze spatial patterns of seismic activity to identify reservoir
  locations
\item
  Examine temporal relationships between deep and shallow seismicity
\item
  Evaluate the multi-reservoir model using observational data
\item
  Contribute to improved eruption forecasting methodologies
\end{itemize}

\bookmarksetup{startatroot}

\chapter{3.Methods and Data}\label{methods-and-data}

\subsection{Seismic Data Acquisition}\label{seismic-data-acquisition}

Earthquake data were obtained from the Instituto Geográfico Nacional
(IGN) web portal, representing publicly available information collected
through a comprehensive network of seismic monitoring stations deployed
across La Palma. The dataset encompasses the critical period from
September 11 to November 9, 2021, capturing pre-eruptive, syn-eruptive,
and post-eruptive phases.

\subsection{Data Processing and Quality
Control}\label{data-processing-and-quality-control}

Raw seismic catalogs were processed using automated web scraping
protocols to ensure reproducibility and systematic data collection.
Quality control measures included verification of event locations,
magnitudes, and depth determinations.

\subsection{Analytical Framework}\label{analytical-framework}

Statistical analysis focused on: - Spatial distribution of hypocenters -
Temporal evolution of seismic activity - Depth-magnitude relationships -
Clustering analysis for reservoir identification

\bookmarksetup{startatroot}

\chapter{3. Results and Analysis}\label{results-and-analysis}

\subsection{Dataset Characteristics}\label{dataset-characteristics}

Analysis of the complete IGN catalog yielded 5,465 seismic events
specifically attributed to La Palma during the study period. These data
were systematically analyzed across multiple dimensions: spatial
distribution, temporal evolution, magnitude characteristics, and depth
clustering patterns.

\subsection{Evidence for Multi-Reservoir
System}\label{evidence-for-multi-reservoir-system}

Our comprehensive analysis reveals three distinct seismic signatures
that strongly support the proposed multi-reservoir magma system:

\subsubsection{Pre-Eruptive Shallow Swarm
Activity}\label{pre-eruptive-shallow-swarm-activity}

Intense earthquake swarms occurred in the shallow subsurface
(\textless{} 10 km depth) during the weeks preceding the September 19th
eruption onset. This activity correlates with significant surface
deformation measurements and indicates shallow magma intrusion
processes.

\subsubsection{Syn- and Post-Eruptive Crustal Reservoir
Activity}\label{syn--and-post-eruptive-crustal-reservoir-activity}

Following eruption initiation, continuous moderate-magnitude seismicity
established at 10-15 km depth, consistent with ongoing magma movement
within the proposed shallow crustal reservoir. This persistent activity
suggests active magma storage and transport processes.

\subsubsection{Deep Mantle Reservoir
Signatures}\label{deep-mantle-reservoir-signatures}

High-magnitude seismic events (M \textgreater{} 4.0) occurred
systematically at 30-40 km depths throughout the study period. These
deeper events, while less frequent than shallow activity, demonstrate
continuous activity within the proposed deep mantle reservoir system.

\bookmarksetup{startatroot}

\chapter{4. Discussion}\label{discussion}

\subsection{Validation of Multi-Reservoir
Models}\label{validation-of-multi-reservoir-models}

The seismic evidence presented strongly validates theoretical
multi-reservoir magma storage models previously proposed for Cumbre
Vieja. The bimodal depth distribution of earthquakes, with distinct
clustering at shallow (10-15 km) and deep (30-40 km) levels, provides
compelling observational support for hierarchical magma storage systems.

\subsection{Implications for Volcanic Hazard
Assessment}\label{implications-for-volcanic-hazard-assessment}

Understanding magma reservoir architecture has direct implications for
eruption forecasting:

\begin{itemize}
\tightlist
\item
  \textbf{Deep reservoir monitoring}: High-magnitude events at mantle
  depths may serve as long-term eruption precursors
\item
  \textbf{Shallow reservoir dynamics}: Swarm activity in the crustal
  reservoir provides short-term eruption warnings
\item
  \textbf{System connectivity}: Temporal relationships between deep and
  shallow activity indicate reservoir interaction
\end{itemize}

\subsection{Methodological Advances}\label{methodological-advances}

This study demonstrates the effectiveness of systematic seismic catalog
analysis for validating geophysical models. The integration of spatial,
temporal, and magnitude characteristics provides a robust framework for
magma system characterization.

\bookmarksetup{startatroot}

\chapter{5. Conclusions}\label{conclusions}

Analysis of 5,465 seismic events from the 2021 La Palma eruption
provides unprecedented insight into active volcanic processes:

\bookmarksetup{startatroot}

\chapter{Key Findings}\label{key-findings}

\begin{tcolorbox}[enhanced jigsaw, arc=.35mm, colback=white, opacityback=0, colframe=quarto-callout-important-color-frame, left=2mm, toprule=.15mm, rightrule=.15mm, breakable, leftrule=.75mm, bottomrule=.15mm]

\begin{enumerate}
\def\labelenumi{\arabic{enumi}.}
\tightlist
\item
  \textbf{Multi-reservoir validation}: Clear evidence supports two-level
  magma storage system
\item
  \textbf{Depth stratification}: Distinct seismic signatures at mantle
  (30-40 km) and crustal (10-15 km) depths
\item
  \textbf{Temporal relationships}: Systematic progression from deep to
  shallow activity preceding eruption
\item
  \textbf{Monitoring implications}: Seismic patterns offer robust
  indicators for eruption forecasting
\end{enumerate}

\end{tcolorbox}

\bookmarksetup{startatroot}

\chapter{Future Directions}\label{future-directions}

\begin{itemize}
\tightlist
\item
  Integration with additional geophysical datasets (deformation, gas
  emissions)
\item
  Development of real-time monitoring algorithms
\item
  Comparative analysis with other volcanic systems
\item
  Enhancement of probabilistic eruption forecasting models
\end{itemize}

\bookmarksetup{startatroot}

\chapter{Data Availability Statement}\label{data-availability-statement}

Seismic data were obtained from the Instituto Geográfico Nacional (IGN)
public database. Data processing scripts, analysis notebooks, and
visualization tools have been developed to ensure reproducibility and
are available through institutional repositories. All methodological
approaches follow open science principles to facilitate community
validation and extension.

\begin{center}\rule{0.5\linewidth}{0.5pt}\end{center}

\bookmarksetup{startatroot}

\chapter{Evaluating Phase Retrieval STEM
Techniques}\label{evaluating-phase-retrieval-stem-techniques}

Information Transfer Analysis in Advanced Electron Microscopy

\hfill\break

\phantomsection\label{abstract}
\bookmarksetup{startatroot}

\chapter{Abstract}\label{abstract-1}

This comprehensive study evaluates advanced methods for phase retrieval
in Scanning Transmission Electron Microscopy (STEM), providing
systematic analysis of information transfer capabilities across multiple
techniques. We examine center-of-mass imaging, tilt-corrected
bright-field STEM, and direct ptychographic methods through rigorous
Contrast Transfer Function (CTF) and Spectral Signal-to-Noise Ratio
(SSNR) analysis. Our evaluation demonstrates that direct ptychographic
methods achieve superior high-frequency performance and uniform
information transfer, while center-of-mass techniques excel at low
spatial frequencies. These findings provide essential guidance for
method selection and experimental optimization in high-resolution
electron microscopy applications.

\bookmarksetup{startatroot}

\chapter{1. Introduction}\label{introduction-1}

Phase retrieval in electron microscopy has emerged as a transformative
approach for achieving atomic-resolution imaging with enhanced contrast
and quantitative information. The fundamental challenge lies in
recovering phase information lost during the detection process, where
only intensity measurements are typically available. STEM techniques
offer unique advantages for phase retrieval applications due to their
focused probe geometry, flexible detector configurations, and ability to
collect comprehensive diffraction data.

Recent advances in computational algorithms and detector technology have
enabled sophisticated phase retrieval methods that can extract
quantitative structural information with unprecedented precision. This
technological convergence has made STEM-based phase retrieval essential
for applications ranging from materials characterization to biological
imaging.

\section{Historical Context and Current
Challenges}\label{historical-context-and-current-challenges}

The development of phase retrieval methods in STEM has evolved through
several generations of techniques, each addressing specific limitations
while introducing new capabilities. Traditional approaches relied on
computational post-processing, while modern methods integrate real-time
reconstruction algorithms with optimized experimental protocols.

\bookmarksetup{startatroot}

\chapter{2. Phase Retrieval Methods in
STEM}\label{phase-retrieval-methods-in-stem}

\section{Theoretical Framework}\label{theoretical-framework}

Our comprehensive evaluation encompasses three primary methodological
approaches, each offering distinct advantages for specific experimental
conditions and applications.

\subsection{Center-of-Mass Imaging}\label{center-of-mass-imaging}

Center-of-mass (COM) imaging represents a computationally efficient
approach that utilizes systematic deflection measurements of the
electron beam to reconstruct phase information. The fundamental
principle relies on measuring the shift in the diffraction pattern
centroid, which directly correlates with local electric fields within
the specimen.

\textbf{Key advantages:} - Real-time processing capabilities - Minimal
computational requirements - Robust performance at low spatial
frequencies

\subsection{Tilt-Corrected Bright-Field
STEM}\label{tilt-corrected-bright-field-stem}

This sophisticated technique addresses aberration-related artifacts by
analyzing intensity variations in bright-field STEM images acquired
across multiple tilt conditions. The method employs advanced algorithms
to separate aberration contributions from genuine specimen-related phase
information.

\textbf{Technical specifications:} - Multi-angle acquisition protocols -
Advanced aberration correction algorithms - Optimized for routine
high-throughput applications

\subsection{Direct Ptychographic
Methods}\label{direct-ptychographic-methods}

Direct ptychography represents the most computationally intensive but
potentially most powerful approach, reconstructing both object and probe
functions simultaneously from overlapping diffraction patterns. This
method offers superior resolution and phase sensitivity through
iterative optimization algorithms.

\textbf{Performance characteristics:} - Highest achievable spatial
resolution - Quantitative phase reconstruction - Optimal for research
applications requiring maximum information content

\bookmarksetup{startatroot}

\chapter{3. Methods and Analysis
Framework}\label{methods-and-analysis-framework}

\subsection{Experimental Design and Data
Collection}\label{experimental-design-and-data-collection}

Our systematic evaluation employed standardized experimental protocols
to ensure meaningful comparisons across all three phase retrieval
methods. Data collection encompassed multiple specimen types and
experimental conditions to assess method robustness and versatility.

\subsection{Analytical Framework}\label{analytical-framework-1}

We developed a comprehensive quantitative framework to evaluate the
information transfer capabilities of each technique through multiple
complementary approaches:

\begin{itemize}
\tightlist
\item
  \textbf{Contrast Transfer Function (CTF)} analysis for spatial
  frequency characterization
\item
  \textbf{Spectral Signal-to-Noise Ratio (SSNR)} calculations for noise
  performance assessment
\item
  \textbf{Phase reconstruction fidelity} metrics for accuracy evaluation
\item
  \textbf{Computational efficiency} benchmarking for practical
  implementation assessment
\end{itemize}

\subsection{Performance Metrics and Evaluation
Criteria}\label{performance-metrics-and-evaluation-criteria}

\subsubsection{Contrast Transfer Function
(CTF)}\label{contrast-transfer-function-ctf}

The CTF describes how different spatial frequencies are transferred from
the object to the image:

\[
CTF(k) = \sin(\chi(k))
\]

where \(\chi(k)\) is the aberration function.

\subsubsection{Spectral Signal-to-Noise Ratio
(SSNR)}\label{spectral-signal-to-noise-ratio-ssnr}

SSNR quantifies the quality of phase information transfer:

\[
SSNR(k) = \frac{|F_{signal}(k)|^2}{|F_{noise}(k)|^2}
\]

\bookmarksetup{startatroot}

\chapter{4. Results and Analysis}\label{results-and-analysis-1}

\subsection{Comprehensive Performance
Assessment}\label{comprehensive-performance-assessment}

Our systematic evaluation across multiple experimental conditions and
specimen types provides definitive guidance for method selection and
optimization strategies.

\subsection{Contrast Transfer Function
Analysis}\label{contrast-transfer-function-analysis}

CTF analysis reveals fundamental differences in spatial frequency
response across the three methods:

\textbf{Direct ptychographic methods} demonstrate the most uniform
information transfer across all spatial frequencies, with consistent
performance from low-frequency structural information to high-frequency
atomic details. This uniformity represents a significant advantage for
quantitative analysis applications.

\textbf{Center-of-mass imaging} exhibits optimal performance at low to
moderate spatial frequencies, making it ideal for applications requiring
rapid structural characterization without atomic-level detail.

\textbf{Tilt-corrected bright-field STEM} provides intermediate
performance with excellent stability across moderate spatial
frequencies, offering the optimal balance for routine analytical
applications.

\subsection{Signal-to-Noise Ratio
Analysis}\label{signal-to-noise-ratio-analysis}

SSNR calculations provide critical insights into practical performance
under realistic experimental conditions:

\begin{enumerate}
\def\labelenumi{\arabic{enumi}.}
\tightlist
\item
  \textbf{Center-of-mass imaging}: Exceptional performance at low
  spatial frequencies with minimal noise amplification
\item
  \textbf{Tilt-corrected bright-field STEM}: Balanced performance across
  all frequencies with robust noise handling
\item
  \textbf{Direct ptychographic methods}: Superior high-frequency
  performance with advanced noise management algorithms
\end{enumerate}

\subsection{Experimental Parameter
Sensitivity}\label{experimental-parameter-sensitivity}

Critical factors affecting performance across all methods include:

\begin{itemize}
\tightlist
\item
  \textbf{Scan sampling effects}: Systematic correlation between
  sampling density and reconstruction quality
\item
  \textbf{Aberration sensitivity}: Direct methods demonstrate superior
  robustness to optical aberrations
\item
  \textbf{Detector geometry optimization}: Performance critically
  dependent on detector configuration for all techniques
\item
  \textbf{Computational resource requirements}: Significant variation in
  processing demands across methods
\end{itemize}

\bookmarksetup{startatroot}

\chapter{5. Discussion}\label{discussion-1}

\subsection{Methodological Advances and
Implications}\label{methodological-advances-and-implications}

Our comprehensive evaluation provides unprecedented insights into the
relative performance characteristics of phase retrieval methods in STEM
applications, establishing a quantitative framework for method selection
and optimization.

\subsection{Advantages of Multi-Metric
Analysis}\label{advantages-of-multi-metric-analysis}

The integration of CTF and SSNR analysis provides superior evaluation
capabilities compared to traditional single-metric approaches:

\begin{itemize}
\tightlist
\item
  \textbf{Comprehensive noise characterization}: Accounts for realistic
  experimental noise conditions
\item
  \textbf{Frequency-dependent performance assessment}: Enables targeted
  optimization for specific applications\\
\item
  \textbf{Quantitative comparison framework}: Facilitates objective
  method selection based on performance requirements
\item
  \textbf{Experimental condition integration}: Incorporates real-world
  constraints and limitations
\end{itemize}

\subsection{Practical Implementation
Guidelines}\label{practical-implementation-guidelines}

\subsubsection{Method Selection
Strategy}\label{method-selection-strategy}

\textbf{For routine analytical applications}: Center-of-mass techniques
provide optimal balance of speed, simplicity, and adequate resolution
for most structural characterization tasks.

\textbf{For advanced research applications}: Direct ptychographic
methods offer superior performance for applications requiring maximum
spatial resolution and quantitative phase information.

\textbf{For high-throughput screening}: Tilt-corrected bright-field STEM
provides optimal compromise between information content and processing
efficiency.

\subsubsection{Experimental Optimization
Framework}\label{experimental-optimization-framework}

Achieving optimal performance across all methods requires systematic
consideration of multiple interdependent factors:

\begin{itemize}
\tightlist
\item
  \textbf{Probe optimization}: Aberration correction and beam conditions
  tailored to specific method requirements
\item
  \textbf{Detector configuration}: Geometry and sensitivity optimized
  for target spatial frequency range
\item
  \textbf{Acquisition parameters}: Scan sampling and integration time
  balanced against noise and drift considerations
\item
  \textbf{Computational infrastructure}: Processing capabilities matched
  to method complexity and throughput requirements
\end{itemize}

\bookmarksetup{startatroot}

\chapter{6. Conclusions}\label{conclusions-1}

Analysis of phase retrieval methods in STEM applications provides
definitive guidance for technique selection and experimental
optimization:

\begin{tcolorbox}[enhanced jigsaw, arc=.35mm, colback=white, opacityback=0, colframe=quarto-callout-important-color-frame, left=2mm, toprule=.15mm, rightrule=.15mm, breakable, leftrule=.75mm, bottomrule=.15mm]

\vspace{-3mm}\textbf{Key Findings}\vspace{3mm}

\begin{enumerate}
\def\labelenumi{\arabic{enumi}.}
\tightlist
\item
  \textbf{Direct ptychographic methods}: Deliver superior high-frequency
  performance and uniform information transfer across all spatial
  frequencies, making them optimal for research applications requiring
  maximum spatial resolution
\item
  \textbf{Multi-metric analysis framework}: CTF and SSNR evaluation
  provides comprehensive assessment capabilities exceeding traditional
  single-metric approaches
\item
  \textbf{Method-specific optimization}: Each technique requires
  tailored experimental protocols and computational resources for
  optimal performance
\item
  \textbf{Application-driven selection}: Method choice should prioritize
  specific experimental requirements over theoretical capabilities
\end{enumerate}

\end{tcolorbox}

\subsection{Future Directions and Research
Opportunities}\label{future-directions-and-research-opportunities}

\textbf{Methodological development}: Integration of hybrid approaches
combining computational efficiency of center-of-mass methods with
resolution capabilities of direct ptychography represents a promising
research direction.

\textbf{Real-time optimization}: Development of adaptive algorithms for
automatic parameter optimization during data acquisition could
significantly improve practical implementation across all methods.

\textbf{Machine learning integration}: Application of advanced machine
learning techniques for noise reduction and reconstruction optimization
offers potential for substantial performance improvements.

\textbf{Standardization efforts}: Establishment of community-wide
benchmarking protocols would facilitate method comparison and accelerate
technique development.

\section{Data Availability
Statement}\label{data-availability-statement-1}

Experimental protocols, analysis algorithms, and benchmark datasets have
been developed following open science principles to facilitate community
validation and method comparison. All computational tools and data
processing workflows are available through institutional repositories
with comprehensive documentation for reproducible implementation.

\section{Acknowledgments}\label{acknowledgments}

We acknowledge the computational resources provided by advanced
microscopy facilities and the valuable scientific discussions with the
international electron microscopy community that informed this
comprehensive analysis. Special recognition is extended to the
instrument developers and algorithm designers whose innovations enabled
this comparative evaluation.

\bookmarksetup{startatroot}

\chapter{3. Evaluación multitemporal de la resiliencia comunitaria ante
eventos
tecnológicos}\label{evaluaciuxf3n-multitemporal-de-la-resiliencia-comunitaria-ante-eventos-tecnoluxf3gicos}

Análisis del desastre de Dosquebradas 2011-2021

\hfill\break

\phantomsection\label{abstract}
\bookmarksetup{startatroot}

\chapter{Resumen}\label{resumen}

La resiliencia comunitaria es la capacidad de una población a enfrentar,
responder y adaptarse a emergencias y desastres manteniendo su
estructura básica, función e identidad. Sin embargo, la resiliencia
comunitaria ante eventos tecnológicos ha sido escasamente estudiada en
Colombia. Aquí se mide y evalúa la resiliencia comunitaria ante
desastres empleando la aplicación de la herramienta ARC-D en una
población del municipio de Dosquebradas (barrio Villa Carola) dónde en
2011 ocurrió un proceso de remoción en masa que desencadenó un derrame
de hidrocarburos generando la pérdida de 33 vidas debido a una explosión
y desbordando las capacidades de respuesta del municipio. Seguimos una
metodología con enfoque cualitativo para la interacción
comunidad-territorio y una con enfoque cuantitativo para el
procesamiento de datos. Para la discusión comunitaria se empleó un
instrumento tipo encuesta medida por grupo focal con facilitador, en la
que se evaluaron cuatro áreas temáticas. La aplicación de la herramienta
tuvo un diagnóstico de resiliencia previo y una prueba piloto de ajuste
al grupo focal. Encontramos un nivel bajo de resiliencia para el año
(2011) con un mayor énfasis en la baja comprensión del riesgo, contrario
al componente más alto que corresponde a cohesión social en el
territorio. El año (2021) registró un nivel de mediana resiliencia con
mejores resultados en el fortalecimiento de la gobernanza para gestionar
el riesgo de desastres, reflejado en su componente destacado, que fue la
participación de las mujeres. Concluimos que en la Comuna 10 de
Dosquebradas mejoró sus capacidades de resiliencia comunitaria ante
desastres por riesgo tecnológico de 30,08\% a 60,18\%, destacando el
aumento de la participación comunitaria, la organización social y la
apropiación de la cultura de la prevención del riesgo. Se recomienda a
los entes territoriales y locales conocer la percepción del riesgo de
las comunidades para enfocar adecuadamente sus políticas, estrategias e
instrumentos en la toma de decisiones.

\textbf{Palabras clave:} Resiliencia comunitaria ante desastres, riesgo
tecnológico, herramienta ARC-D, gestión del riesgo de desastres,
análisis multitemporal de resiliencia, percepción del riesgo.

\bookmarksetup{startatroot}

\chapter{1. Introducción}\label{introducciuxf3n}

La resiliencia comunitaria es la capacidad de las comunidades y hogares
que viven dentro de sistemas complejos para anticiparse y adaptarse a
los riesgos, para absorber, responder y recuperarse de las amenazas y
estresores de una manera oportuna y efectiva sin comprometer sus
posibilidades a largo plazo, mejorando finalmente su bienestar {[}4{]}.
Este definición sigue la del grupo de trabajo intergubernamental de
expertos de composición abierta sobre los indicadores mundiales para las
metas mundiales del Marco de Sendai que define la resiliencia como ``la
capacidad que tiene un sistema, una comunidad o una sociedad expuestos a
una amenaza para resistir, absorber, adaptarse, transformarse y
recuperarse de sus efectos de manera oportuna y eficiente, en particular
mediante la preservación y la restauración de sus estructuras y
funciones básicas por conducto de la gestión de riesgos'' {[}3{]}. Otras
aproximaciones definen la resiliencia comunitaria como la capacidad de
los sistemas para hacer frente y adaptarse a eventos extremos, sin
perder su estructura básica, y hace énfasis en que la cultura de la
resiliencia permite reducir pérdidas humanas, socioeconómicas y
ambientales {[}2{]}\\
La resiliencia comunitaria tiene dos componentes. Primero, la capacidad
de adaptación, que se define como las condiciones que permiten a las
personas y/o comunidades anticipar y responder al cambio, minimizar las
consecuencias, recuperarse y aprovechar las nuevas oportunidades
{[}12{]}. Segundo, la percepción del riesgo, en el cual las comunidades
perciben el riesgo basado en procesos cognitivos influidos por diversos
factores, tales como las características y la gravedad del riesgo,
experiencias previas, la cantidad y calidad de la información
disponible, así como los valores individuales y sociales y el
conocimiento sobre el fenómeno que genera el riesgo {[}14{]}.\\
La importancia de considerar la percepción del riesgo de desastres como
una necesidad reciente, especialmente en el contexto del cambio
climático y el incremento de desastres, ha sido abordado en análisis de
percepción del riesgo, a partir de métodos psicosociales sustentados en
la subjetividad de los seres humanos ante los escenarios de riesgo
presentes en sus entornos {[}15{]}. Sin embargo, los enfoques se han
direccionado a eventos de tipo natural y socio-natural dejando a un lado
los escenarios de riesgo antrópico-tecnológicos. Los riesgos
tecnológicos en Sudamérica han sido poco explorados a partir de
ejercicios de percepción del riesgo que permitan establecer niveles de
resiliencia comunitaria ante este tipo de eventos.\\
Los autores Sandoval, Navarrete, \& Cuadra {[}14{]} resaltan que la
resiliencia comunitaria se ha estudiado en Latinoamérica principalmente
en países como Chile, seguido de Puerto Rico y Brasil. Los principales
riesgos estudiados son terremotos (12), inundaciones (11) y tsunamis
(7), y los escasamente estudiados son sequías, erupciones volcánicas y
eventos extremos vinculados al cambio climático {[}13{]}. Los autores
afirman que en Latinoamérica una de las principales barreras para el
desarrollo o despliegue de la resiliencia comunitaria, son las
condiciones de vulnerabilidad en las dimensiones políticas, sociales,
económicas y ambientales. Frente a este panorama, en Colombia la
resiliencia comunitaria ha sido escasamente estudiada y los trabajos
realizados se han enfocado en análisis comparativos a partir de índices
compuestos de resiliencia para dimensiones ecosistémicas. Estos trabajos
han utilizado variables como diversidad de suelos, diversidad de
empresas, índice de desempeño integral y capacidad de carga sobre el
ecosistema, arrojando como resultados gran disparidad entre los niveles
de resiliencia de los entes territoriales estudiados {[}17{]}.\\
Aquí presentamos la evaluación de la resiliencia comunitaria
multitemporal entre los años 2011 y 2021 el municipio de Dosquebradas en
el barrio Villa Carola, dónde en 2011 ocurrió un evento de origen
natural que desencadenó un accidente tecnológico o Natech, debido a un
proceso de remoción en masa que causó el derrame de hidrocarburos
ocasionando 33 muertes debido a una explosión por el hidrocarburo
derramado en un punto caliente y desbordando las capacidades de
respuesta municipales.

Actualmente, la Comuna 10 donde se ubica el barrio Villa Carola,
presenta múltiples amenazas socio-naturales por deslizamientos y
amenazas tecnológicas por actividades industriales, transporte de
hidrocarburos a partir de infraestructura como poliductos y gasoductos,
transporte de sustancias peligrosas a partir de vehículos cisterna,
aglomeración de público, entre otras.

\begin{tcolorbox}[enhanced jigsaw, arc=.35mm, colback=white, opacityback=0, colframe=quarto-callout-important-color-frame, left=2mm, toprule=.15mm, rightrule=.15mm, breakable, leftrule=.75mm, bottomrule=.15mm]

\vspace{-3mm}\textbf{Caso de Estudio: Desastre tecnológico en la Comuna 10 de Dosquebradas
(2011)}\vspace{3mm}

El 23 de diciembre de 2011 ocurrió en la Comuna 10 de Dosquebradas una
de las tragedias más graves asociadas a riesgo tecnológico en Colombia.
Una fuga de combustible en el poliducto Puerto Salgar--Cartago descendió
por la quebrada Aguazul y al entrar en contacto con una fuente de calor,
desencadenó una serie de explosiones. El evento afectó principalmente a
los barrios Villa Carola, La Divisa, La Romelia en la Comuna 10 del
municipio, dejando 33 personas fallecidas, 110 heridas, más de cien
viviendas afectadas y 38 destruidas, además de pérdidas en
establecimientos comerciales.

\pandocbounded{\includegraphics[keepaspectratio]{index_files/mediabag/PZrfb8X8BO558hkbXYiE.png}}

\emph{Fuente: Archivo Gobernación de Risaralda, 2011.}

Las causas se atribuyeron tanto a factores antrópicos, como la presencia
de válvulas ilegales en el poliducto, como a factores naturales, en
particular la reptación del terreno producto de la ola invernal de 2011,
que generó tensiones y la fractura de la tubería. El desastre también
impactó ambientalmente la quebrada Aguazul, contaminada con más de 1,400
galones de gasolina que afectaron su ronda hídrica y la bocatoma que
abastecía a cerca de 25,000 personas.

La atención inmediata contó con la participación de la comunidad,
organismos de socorro, autoridades locales y nacionales, así como la
empresa Ecopetrol. Posteriormente, se desarrollaron procesos de diálogo
y reparación, que permitieron una recuperación colectiva menos
traumática y consolidaron aprendizajes comunitarios sobre la gestión del
riesgo. El Desastre presentado en el barrio Villa Carola de Dosquebradas
fue documentado en el libro \emph{Huellas de Esperanza}, en el que se
narran los eventos ocurridos en diciembre de 2011 y en los meses
siguientes después de la tragedia en las etapas de rehabilitación y
recuperación postdesastre.

\emph{Fuente: adaptado de Ecopetrol S.A., Veeduría Ciudadana \&
Fundación Social Cooplarosa (2015). Huellas de Esperanza.}

\end{tcolorbox}

\bookmarksetup{startatroot}

\chapter{2. Metodología}\label{metodologuxeda}

\section{Municipio de Dosquebradas y contexto de la Comuna
10}\label{municipio-de-dosquebradas-y-contexto-de-la-comuna-10}

El municipio de Dosquebradas pertenece al departamento de Risaralda en
Colombia. Se encuentra en el sector oriental del departamento y se ubica
en la parte occidental de la Cordillera Central. Sus límites municipales
son al norte y al este con el municipio de Santa Rosa, al sur con la
ciudad de Pereira, y al noroeste con el municipio de Marsella.
Dosquebradas se compone de su casco urbano con más de trescientos
barrios, dos centros poblados y veinticinco veredas. La población es de
217,178 habitantes, distribuidos en 77,387 unidades de vivienda y 64,576
hogares. La proyección poblacional ajustada indica que el 94.6\%
pertenece a la cabecera municipal, mientras que el 5.4\% pertenece a los
centros poblados y rurales dispersos.\\
En el mapa de la \textbf{?@fig-1} se visualiza la localización de la
Comuna 10 del municipio de Dosquebradas y la distribución espacial del
estado de la comuna frente a escenarios de amenaza o riesgo, tanto de
tipo tecnológico como socio-natural, identificados en el territorio. Se
observan las amenazas tecnológicas, entre ellas las áreas de
aglomeración, las rutas de transporte de carga y los sectores de
almacenamiento de carga, en el contexto de la dinámica urbana y social
de la comuna. Además, zonas con amenaza por movimientos en masa, red
hidrográfica y líneas de tensión eléctrica que atraviesan el área
también se diferencian en sectores clasificados como de alto riesgo
mitigable y alto riesgo no mitigable por fenómenos recurrentes. Los
últimos permiten dar contexto frente a los eventos de tipo Natech que
puedan presentarse en la comuna, como los registrados en el evento de la
tragedia del año 2011. Se destaca la ausencia del poliducto en la
comuna, dado que dicha infraestructura suspendió sus servicios tiempo
atrás a la aplicación del presente ejercicio.\\
\pandocbounded{\includegraphics[keepaspectratio]{index_files/mediabag/KDkCnAm5IpmZIpmZIfhE.png}}\\
\textbf{Figura 1.} Mapa de localización del municipio de Dosquebradas
(fuente: elaboración propia).

\section{Herramienta ARC-D para evaluación de resiliencia
comunitaria}\label{herramienta-arc-d-para-evaluaciuxf3n-de-resiliencia-comunitaria}

Se utilizó la herramienta ARC-D para medir la resiliencia comunitaria
ante desastres desarrollada por la organización GOAL
(\url{https://www.goalglobal.org/}). Esta herramienta tiene un enfoque
mixto, con elementos cualitativos y cuantitativos. El enfoque
cualitativo hace referencia a la interacción con la comunidad de los
territorios, con los cuales se recoge la información desde una
perspectiva social y comunitaria, a través de un instrumento tipo
encuesta de 30 preguntas (\textbf{?@fig-2}), las cuales cuentan con
instrumentos de verificación. El enfoque cuantitativo se refiere al
procesamiento de los datos y permite obtener gráficos de red para la
sistematización y presentación de la información de tipo numérico y
jerárquico. La herramienta cuenta con dos etapas. Una etapa A que
realiza un diagnóstico general de la comunidad y una etapa B, que se
enfoca en las preguntas de evaluación en el grupo focal
(\textbf{?@fig-2}). De acuerdo con GOAL {[}5{]}, esta herramienta se
construye de acuerdo al trabajo en resiliencia ante desastres.

\begin{tcolorbox}[enhanced jigsaw, colback=white, opacityback=0, bottomtitle=1mm, toprule=.15mm, colbacktitle=quarto-callout-note-color!10!white, titlerule=0mm, toptitle=1mm, breakable, leftrule=.75mm, coltitle=black, arc=.35mm, opacitybacktitle=0.6, colframe=quarto-callout-note-color-frame, left=2mm, title=\textcolor{quarto-callout-note-color}{\faInfo}\hspace{0.5em}{Etapas de la herramienta ARC-D de GOAL}, bottomrule=.15mm, rightrule=.15mm]

\pandocbounded{\includegraphics[keepaspectratio]{index_files/mediabag/gcMPxsNmcKQNAAAAABJR.png}}

\textbf{Figura 2.} Componentes herramienta ARC-D. La herramienta de
evaluación de resiliencia comunitaria establecida por GOAL evalúa cuatro
áreas temáticas que conforman 30 componentes. La evaluación produce
cinco niveles o categorías que describen condiciones desde mínima hasta
alta resiliencia, de acuerdo con la percepción del grupo focal evaluado.
(Fuente: Autores).

\subsection{Categorías de evaluación de la herramienta ARC-D de
GOAL}\label{categoruxedas-de-evaluaciuxf3n-de-la-herramienta-arc-d-de-goal}

\begin{longtable}[]{@{}
  >{\centering\arraybackslash}p{(\linewidth - 6\tabcolsep) * \real{0.2639}}
  >{\centering\arraybackslash}p{(\linewidth - 6\tabcolsep) * \real{0.2639}}
  >{\raggedright\arraybackslash}p{(\linewidth - 6\tabcolsep) * \real{0.2361}}
  >{\raggedright\arraybackslash}p{(\linewidth - 6\tabcolsep) * \real{0.2361}}@{}}
\caption{Categorías de evaluación de la herramienta ARC-D de
GOAL}\label{tbl-1}\tabularnewline
\toprule\noalign{}
\begin{minipage}[b]{\linewidth}\centering
\%
\end{minipage} & \begin{minipage}[b]{\linewidth}\centering
Nivel
\end{minipage} & \begin{minipage}[b]{\linewidth}\raggedright
Categoría
\end{minipage} & \begin{minipage}[b]{\linewidth}\raggedright
Descripción
\end{minipage} \\
\midrule\noalign{}
\endfirsthead
\toprule\noalign{}
\begin{minipage}[b]{\linewidth}\centering
\%
\end{minipage} & \begin{minipage}[b]{\linewidth}\centering
Nivel
\end{minipage} & \begin{minipage}[b]{\linewidth}\raggedright
Categoría
\end{minipage} & \begin{minipage}[b]{\linewidth}\raggedright
Descripción
\end{minipage} \\
\midrule\noalign{}
\endhead
\bottomrule\noalign{}
\endlastfoot
0-20 & 1 & Mínima Resiliencia & Poca conciencia del problema o poca
motivación para abordarlo. Acciones limitadas a respuestas durante
crisis. \\
21-40 & 2 & Baja Resiliencia & Se tiene conciencia del problema, se
cuenta con capacidad para actuar, pero de manera limitada, con
intervenciones fragmentadas y a corto plazo \\
41-60 & 3 & Mediana Resiliencia & Desarrollo e implementación de
soluciones. Capacidad de actuar es mejorada y sustancial. Intervenciones
numerosas y de largo plazo \\
61-80 & 4 & Resiliencia & Coherencia e integración. Intervenciones
amplias, cubriendo los mayores aspectos del problema y ligadas a una
estrategia coherente y de largo plazo. \\
81-100 & 5 & Alta Resiliencia & Existe una cultura de seguridad en los
actores en donde la gestión del riesgo en toda política, planeación,
prácticas, actitudes y comportamientos. \\
\end{longtable}

\subsection{Glosario de términos relacionados con la herramienta
ARC-D}\label{glosario-de-tuxe9rminos-relacionados-con-la-herramienta-arc-d}

\textbf{Capacidad:} Es la habilidad de las personas, instituciones y
sociedades para llevar a cabo funciones, resolver problemas y fijarse y
obtener objetivos. De acuerdo con la UNISDR, es la combinación de todas
las fortalezas, atributos y recursos disponibles dentro de una
comunidad, sociedad u organización para poder obtener las metas
acordadas. La capacidad puede incluir infraestructura y medios físicos,
instituciones, habilidades de adaptación de las sociedades, así como
conocimiento humano, habilidades y atributos colectivos tales como las
relaciones sociales, liderazgo, y administración. La evaluación de
capacidades es un término para el proceso mediante el cual la capacidad
de un grupo es revisada contra sus metas deseadas, y las brechas de
capacidad son identificadas para acciones futuras.

\textbf{Estresores:} Los estresores son tendencias a largo plazo que
socavan el potencial de un sistema o proceso y que aumentan la
vulnerabilidad de los actores dentro de ella. Estos pueden ser
degradación de los recursos naturales, pérdida en la producción
agrícola, urbanización, cambios demográficos, cambio climático,
inestabilidad política y reducción en los ingresos.

\textbf{Gobernanza:} La gobernanza es el proceso de toma de decisiones y
la subsecuente implementación (o no implementación) de esas decisiones.
Es el ejercicio de la autoridad política, económica y administrativa en
el manejo de los asuntos de un país en todos los niveles. Se compone de
mecanismos, procesos e instituciones a través de los cuales los
ciudadanos y grupos articulan sus intereses, ejercen sus derechos
legales, cumplen sus obligaciones y median sus diferencias. La
gobernanza abarca, pero también trasciende el estado. Abarca a todos los
grupos relevantes incluyendo al sector privado y organizaciones de la
sociedad civil.

\end{tcolorbox}

\subsection{Componentes evaluativos del
ARC-D}\label{componentes-evaluativos-del-arc-d}

La herramienta ARC-D estructura la resiliencia comunitaria en cuatro
áreas temáticas que responden a los objetivos del marco de Sendai, que
permiten analizar de manera amplia las capacidades de una comunidad
frente a situaciones de desastre.\\
La primera área temática corresponde a comprender el riesgo de
desastres, la cual busca identificar en qué medida la comunidad reconoce
y se apropia del conocimiento sobre sus amenazas, así como de las
estrategias de sensibilización y educación. Se agrupa en cuatro
componentes: evaluación participativa y evaluación científica del
riesgo, la difusión de información y la educación de los niños en
reducción de riesgo. La segunda área temática se orienta a fortalecer la
gobernanza para gestionar el riesgo de desastres y reúne siete
componentes. En este caso, los componentes que se abordan son los
mecanismos de planificación del desarrollo y planificación territorial,
la toma de decisiones colectivas, la inclusión de grupos vulnerables y
de mujeres, el conocimiento de derechos, así como las alianzas para la
reducción de riesgo y los procesos de recuperación.\\
La tercera área temática corresponde a reducir la vulnerabilidad a
desastres para mejorar la resiliencia que contempla doce componentes: la
gestión ambiental sostenible, la seguridad y gestión del agua, el acceso
y conciencia de la salud, el suministro seguro de alimentos, las
prácticas de medios de vida resistentes a amenazas, el acceso a mercado,
el acceso a servicios financieros, la protección de ingresos y activos,
el acceso a protección social, la cohesión social y prevención de
conflictos, la infraestructura crítica y la vivienda.\\
Finalmente, la cuarta área temática se dedica a mejorar la preparación
ante desastres para lograr respuestas efectivas y promover la
reconstrucción resiliente después de una emergencia y se conformada por
siete componentes que abarcan: la planificación de contingencia y
recuperación, los sistemas de alerta temprana, la capacidad de
preparación y respuesta, los servicios de salud y educación durante
emergencias, la infraestructura en contextos de crisis y el liderazgo
comunitario en la respuesta y la recuperación.\\
Los componentes de las áreas temáticas de la evaluación de resiliencia
comunitaria se evalúan según los parámetros de resiliencia establecidos
por GOAL. A continuación se presentan los términos relacionados con el
riesgo tecnológico, los cuales se mencionan en el Plan Municipal de
Gestión del Riesgo de Desastres en su componente programático.

\begin{tcolorbox}[enhanced jigsaw, colback=white, opacityback=0, bottomtitle=1mm, toprule=.15mm, colbacktitle=quarto-callout-note-color!10!white, titlerule=0mm, toptitle=1mm, breakable, leftrule=.75mm, coltitle=black, arc=.35mm, opacitybacktitle=0.6, colframe=quarto-callout-note-color-frame, left=2mm, title=\textcolor{quarto-callout-note-color}{\faInfo}\hspace{0.5em}{Glosario de términos relacionados con el riesgo tecnológico}, bottomrule=.15mm, rightrule=.15mm]

\textbf{Accidente tecnológico:} eventos generados por el uso y acceso a
la tecnología, originados por eventos antrópicos, naturales,
socio-naturales y propios de la operación. Comprende fugas, derrames,
incendios y explosiones asociados a la liberación súbita de sustancias
y/o energías con características de peligrosidad. Usualmente, se suelen
asociar los accidentes tecnológicos exclusivamente con las instalaciones
industriales o equipamientos de alta tecnología. No obstante, la
experiencia de accidentabilidad deja entrever muchos eventos en el
sector residencial y a nivel de obras civiles.

\textbf{Amenaza tecnológica:} Amenaza relacionada con accidentes
tecnológicos o industriales, procedimientos peligrosos, fallos de
infraestructura o de ciertas actividades humanas, que pueden causar
muerte o lesiones, daños materiales, interrupción de la actividad social
y económica o degradación ambiental. Algunas veces llamadas amenazas
antropogénicas. Ejemplos incluyen contaminación industrial, descargas
nucleares y radioactividad, desechos tóxicos, ruptura de presas,
explosiones e incendios.

\textbf{Riesgo tecnológico:} Daños o pérdidas potenciales que pueden
presentarse debido a los eventos generados por el uso y acceso a la
tecnología, originados en sucesos antrópicos, naturales, socio-naturales
y propios de la operación.

\end{tcolorbox}

\bookmarksetup{startatroot}

\chapter{3. Recolección y Procesamiento de
Datos}\label{recolecciuxf3n-y-procesamiento-de-datos}

\section{Reconocimiento de la Comuna 10 y sus
líderes}\label{reconocimiento-de-la-comuna-10-y-sus-luxedderes}

La fase exploratoria consistió en actividades encaminadas a la revisión
de información bibliográfica de artículos, publicaciones y metodologías
relacionadas con resiliencia comunitaria, riesgo tecnológico, percepción
del riesgo e instrumentos de planificación de gestión del riesgo de
orden local. La revisión de la herramienta ARC-D con sus componentes y
glosario (\textbf{Caja 2}) y la participación en capacitaciones
virtuales sobre la herramienta dirigidas por GOAL sede Nicaragua.\\
Previo a la aplicación de la herramienta, se realizó el acercamiento al
territorio y la comunidad mediante una visita de campo, tres reuniones
preliminares virtuales y una reunión presencial con los líderes
comunitarios para socializar el objetivo de la aplicación de la
herramienta ARC-D, de donde surgieron propuestas de la comunidad,
orientadas a la aplicación del grupo focal, correspondiente a la
evaluación de los 30 componentes del ARC-D. Igualmente, se efectuó una
prueba piloto que permitió ajustar la herramienta de manera conceptual y
metodológica, generando adaptaciones a los formatos de aplicación y a
una mejor contextualización del ejercicio, lo que facilitó su aplicación
final con los líderes de la Comuna 10. Aunque la comunidad ya tenía
conocimientos sobre los escenarios de amenaza y riesgo de su comuna, se
acordó construir una línea base previa sobre la terminología a utilizar
antes de aplicar la herramienta.\\
El riesgo considerado en la Comuna 10 fue ante eventos tecnológicos,
dados los antecedentes de la explosión de Villa Carola en el año 2011 y
la presencia de zonas potenciales de riesgo tecnológico en varias zonas
de la comuna. Esto, a su vez, permitió establecer el criterio de
multitemporalidad en la medición de la resiliencia comunitaria, algo
innovador en la aplicación del ARC-D.

\section{Aplicación de la herramienta ARC-D en el grupo
focal}\label{aplicaciuxf3n-de-la-herramienta-arc-d-en-el-grupo-focal}

Durante octubre de 2021, se llevó a cabo en el barrio Villa Carola del
municipio de Dosquebradas la sesión de grupo focal con 14 líderes
comunitarios de diferentes barrios de la Comuna 10, en donde se
evaluaron los 30 componentes de la herramienta ARC-D.\\
El desarrollo de la actividad consistió en la evaluación por la
comunidad de la percepción de riesgo frente a escenarios de riesgo de
tipo tecnológico, entendiendo los antecedentes de emergencias y
desastres ocurridos en la C por este tipo de eventos, específicamente el
evento del año 2011. Con base en lo anterior, y dado que la comunidad
había presenciado el evento de 2011, así como la etapa de rehabilitación
y reconstrucción del barrio Villa Carola, como testigos directos e
indirectos de los procesos institucionales y comunitarios que se dieron,
se propuso a los asistentes que la evaluación de la aplicación de la
herramienta ARC-D se dividiera en dos momentos. El primer momento era la
evaluación en retrospectiva de cómo se percibían los componentes
evaluados del 2011 hacia atrás, y un segundo momento, en el que se
evaluara la percepción actual de riesgo (año 2021) considerando los
mismos 30 componentes del ARC-D frente a eventos tecnológicos. De esta
manera se evaluaría la multitemporalidad de la percepción del riesgo y
se identificarían las capacidades de resiliencia comunitaria frente a
escenarios de tipo tecnológico materializados en la comunidad.\\
\pandocbounded{\includegraphics[keepaspectratio]{index_files/mediabag/UJyOgOkh1wcJlKiuntkW.png}}\\
\textbf{Figura 3.} Aplicación de la herramienta ARC-D con grupo focal,
líderes de la Comuna 10 el 09 de octubre de 2021 (Fuente: Autores).

\textbf{2.3 Análisis de datos}

Se analizó la información obtenida en campo y en el grupo focal, la cual
consistió en las siguientes actividades. La primera parte consiste en el
análisis de las encuestas, el cual se realizó mediante el procesamiento
de la información en el software CommCare®, donde se registraron los
datos y posteriormente se integraron a la plataforma de GOAL. El
diligenciamiento de la información permitió obtener el registro de la
Comuna 10 en la plataforma Nexus, donde la información fue representada
a través de gráficos de red que pueden consultarse en la página web de
GOAL Resilience Nexus
(\url{https://resiliencenexus.org/es/global_scores/all-scores}/) en el
radar de resiliencia, seleccionando la siguiente ruta donde se
encuentran la evaluación 1 y 2, correspondientes al año 2011 y 2021
respectivamente (ruta: Colombia \textgreater{} Risaralda \textgreater{}
Comuna 10 \textgreater{} 1 o 2 (según evaluación que desee consultar).\\
Se consideró el análisis de entrevistas previas, y se examinó la
información cualitativa aportada por los participantes, lo que permitió
contextualizar los resultados de las encuestas y profundizar en las
percepciones de la comunidad sobre los diferentes componentes de la
resiliencia, útiles en el análisis y comprensión de la comunidad frente
a los riesgos tecnológicos presentados en su localidad.\\
Posteriormente, se realizó el análisis comparativo de resultados de la
evaluación, considerando cada componente de la herramienta y
estableciendo un contraste temporal entre los años 2011 antes de la
explosión (año del evento de la explosión en Villa Carola) y 2021
después de la explosión y de los procesos de recuperación y
reconstrucción (año de aplicación de la herramienta ARC-D), con el fin
de identificar los cambios, las mejoras y los aspectos que requieren
mayor intervención en las capacidades de resiliencia de la comunidad
frente a eventos tecnológicos.\\
Finalmente, se emitió la calificación de la resiliencia comunitaria
basada en la aplicación de la herramienta ARC-D con enfoque
multitemporal en eventos de tipo tecnológico, y se elaboró una propuesta
de estrategias orientadas a mejorar las condiciones de resiliencia de la
Comuna 10 en el municipio de Dosquebradas.

\begin{longtable}[]{@{}
  >{\centering\arraybackslash}p{(\linewidth - 8\tabcolsep) * \real{0.2000}}
  >{\raggedright\arraybackslash}p{(\linewidth - 8\tabcolsep) * \real{0.2000}}
  >{\raggedright\arraybackslash}p{(\linewidth - 8\tabcolsep) * \real{0.2000}}
  >{\centering\arraybackslash}p{(\linewidth - 8\tabcolsep) * \real{0.2000}}
  >{\centering\arraybackslash}p{(\linewidth - 8\tabcolsep) * \real{0.2000}}@{}}
\toprule\noalign{}
\begin{minipage}[b]{\linewidth}\centering
Proceso metodológico para la evaluación de la resiliencia comunitaria
\end{minipage} & \begin{minipage}[b]{\linewidth}\raggedright
\end{minipage} & \begin{minipage}[b]{\linewidth}\raggedright
\end{minipage} & \begin{minipage}[b]{\linewidth}\centering
\end{minipage} & \begin{minipage}[b]{\linewidth}\centering
\end{minipage} \\
\midrule\noalign{}
\endhead
\bottomrule\noalign{}
\endlastfoot
\textbf{Acción} & \textbf{Actividad} & \textbf{Producto o resultado
esperado} & \textbf{Participantes} & \textbf{Medio} \\
Reconocimiento de la Comuna 10 y sus líderes & Capacitación con GOAL
para la aplicación de la herramienta ARC - D & Entendimiento de la
herramienta para la medición de resiliencia comunitaria & Personal de
GOAL & Virtual \\
& Acercamiento con líderes de la Comuna 10 del municipio de Dosquebradas
& Contextualización del territorio y de los estresores presentados en la
Comuna 10 & Líderes comunitarios de gestión del riesgo de Dosquebradas &
Presencial \\
& Prueba piloto con grupo focal preliminar & Adaptación de la
herramienta ARC - D al contexto comunitario de Dosquebradas frente a
eventos de tipo tecnológico & Líderes comunitarios de la Comuna 10 de
Dosquebradas & Mixto \\
Aplicación de la herramienta ARC - D en el grupo focal & Evaluación de
resiliencia comunitaria frente a eventos tecnológicos en el contexto
temporal del año 2011 (Antes de la explosión). & Resultados de
percepción del riesgo ante eventos tecnológicos para la medición de
resiliencia comunitaria después de la explosión - Temporalidad 2021. &
Grupo focal con los líderes comunitarios de la Comuna 10 de Dosquebradas
& Presencial \\
& Evaluación de resiliencia comunitaria frente a eventos tecnológicos en
el contexto temporal del año 2021 (Después de la explosión). &
Resultados de percepción del riesgo ante eventos tecnológicos para la
medición de resiliencia comunitaria antes de la explosión - Temporalidad
2011. & Grupo focal con los líderes comunitarios de la Comuna 10 de
Dosquebradas & Presencial \\
Análisis de datos y resultados & Procesamiento de información en
CommCare & Registro de las evaluaciones de la Comuna 10 de Dosquebradas
en la plataforma Nexus Resilience de GOAL. & Plataforma CommCare &
Virtual \\
& Análisis de entrevistas previas & Contextualización de los resultados
obtenidos en la aplicación de ARC - D. & Autores & - \\
& Análisis multitemporal de las evaluaciones del ARC - D & Comparativo
de los resultados de percepción comunitaria frente a eventos
tecnológicos antes y después de la explosión de Villa Carola en la
Comuna 10 & Autores & - \\
& Resultados de la aplicación de la herramienta & Propuesta de
estrategias orientadas a mejorar las condiciones de resiliencia
comunitaria de la comuna 10 de Dosquebradas & Autores & - \\
\end{longtable}

\textbf{2.4 Desarrollo de la aplicación de la herramienta ARC-D en el
grupo focal}

Para iniciar la actividad, se realizó una introducción con los líderes
comunitarios de la Comuna 10, Diego Buitrago y Elmer Castañeda, quienes
realizaron la presentación al espacio y socializaron el objetivo de la
sesión. Paralelamente, se realizó una contextualización, donde se
explicó el alcance y el contexto de resiliencia comunitaria y el
escenario de riesgo tecnológico. En la \textbf{?@fig-4} se presenta el
registro fotográfico de la introducción.

\pandocbounded{\includegraphics[keepaspectratio]{index_files/mediabag/AOIyOybLZ1x6AAAAAElF.png}}\\
\textbf{Figura 4.} Introducción al grupo focal el sábado 09 de octubre
de 2021 (Fuente: Autores).

\textbf{Evaluación.} Se organizó el grupo focal y, con una asistencia de
14 personas (5 hombres y 9 mujeres), se reconocieron como líderes y
lideresas de la Comuna 10. Las personas se escogieron por su
conocimiento del territorio, su trabajo con las comunidades y su
experiencia en el desastre del año 2011, por lo que su información es
contundente y especializada para dicha evaluación. Se pactaron acuerdos
para desarrollar la sesión y se explicó que la calificación es
multitemporal, con un contexto de referencia del año 2011 y otro para
las condiciones del año 2021. Con estas claridades se realizó la
evaluación de la resiliencia comunitaria ante desastres en la Comuna 10.
En la \textbf{?@fig-5} se evidencia el registro fotográfico de los
participantes\textbf{.}

\textbf{Cierre}. Se realizó un balance de la metodología aplicada y se
indagó en la percepción que se tuvo del ejercicio. Se manifestó que es
relevante la evaluación diez años después del desastre, pues permite
evaluar sus capacidades a lo largo del tiempo. A su vez, se analizaron
de forma retrospectiva aspectos que han mejorado en términos de
capacidades comunitarias y otros en los que se requiere un mayor
esfuerzo. Finaliza la sesión con palabras de las comunidades y la
propuesta de seguir generando estos espacios por parte de la academia.

\pandocbounded{\includegraphics[keepaspectratio]{index_files/mediabag/E6CeL07EkUEdAAAAAElF.png}}\\
\textbf{Figura 5.} Cierre del grupo focal Comuna 10 el 9 de octubre de
2021 (Fuente: Autores).

\section{Caracterización del riesgo tecnológico en el municipio de
Dosquebradas}\label{caracterizaciuxf3n-del-riesgo-tecnoluxf3gico-en-el-municipio-de-dosquebradas}

La caracterización de riesgos tecnológicos en el municipio de
Dosquebradas se identifican sectorialmente en el Plan Municipal de
Gestión del Riesgo de Desastres del municipio (2023) por comunas de la
siguiente manera (Table~\ref{tbl-2}). Esta caracterización se basa en la
información disponible en los instrumentos de planificación del
municipio de Dosquebradas.

\begin{longtable}[]{@{}
  >{\raggedright\arraybackslash}p{(\linewidth - 2\tabcolsep) * \real{0.5000}}
  >{\raggedright\arraybackslash}p{(\linewidth - 2\tabcolsep) * \real{0.5000}}@{}}
\caption{Riesgos tecnológicos asociados por comuna para el municipio de
Dosquebradas}\label{tbl-2}\tabularnewline
\toprule\noalign{}
\begin{minipage}[b]{\linewidth}\raggedright
Comuna o Corregimiento
\end{minipage} & \begin{minipage}[b]{\linewidth}\raggedright
Riesgo tecnológico asociado
\end{minipage} \\
\midrule\noalign{}
\endfirsthead
\toprule\noalign{}
\begin{minipage}[b]{\linewidth}\raggedright
Comuna o Corregimiento
\end{minipage} & \begin{minipage}[b]{\linewidth}\raggedright
Riesgo tecnológico asociado
\end{minipage} \\
\midrule\noalign{}
\endhead
\bottomrule\noalign{}
\endlastfoot
Comuna 10 (Villa Carola) & Estaciones de servicio, estación de
reabastecimiento de energía con entradas y salidas de alta tensión,
sector industrial, circulación de hidrocarburos y derivados (Actualmente
la línea adscrita a TGI y antiguamente la línea adscrita a Ecopetrol y
con un evento significativo en materia de riesgos tecnológicos). \\
Alto del Nudo & Antiguamente la red Puerto Salgar-Cartago adscrita a
Ecopetrol. Tuberías de conducción. Sector industrial, comercial y
particulares. \\
\end{longtable}

\section{Condiciones de resiliencia comunitaria ante desastres por
riesgo
tecnológico}\label{condiciones-de-resiliencia-comunitaria-ante-desastres-por-riesgo-tecnoluxf3gico}

Para realizar la evaluación de los componentes de la resiliencia en la
Comuna 10, se realizaron reuniones preliminares con líderes de la Comuna
10, los cuales generaron retroalimentación a las preguntas de la
herramienta. Esta reunión se denomina, según la herramienta ARC-D,
``Grupo Focal''. Las condiciones comunitarias de la Comuna 10 de
Dosquebradas incluyen barrios, organizaciones comunitarias y
características económicas (\textbf{?@fig-6}).

\pandocbounded{\includegraphics[keepaspectratio]{index_files/mediabag/x-jyl1KE3uJdQAAAABJR.png}}

\textbf{Figura 6.} Características comunitarias de la Comuna 10 de
Dosquebradas. La conforman los barrios: Carlos Ariel Escobar, La Romelia
Alta y Baja, La Divisa, Galaxia, Las Acacias, Los Pinos, Los Guamos, El
Bosque Carbonero, La Floresta, Estación Gutiérrez, Villa Carola, Bosques
de la Acuarela, Lara Bonilla, El Rosal, El Chicó, Villa Colombia, La
Semilla, Tejares de la Loma, Nuevo Bosque.

\bookmarksetup{startatroot}

\chapter{4. Resultados}\label{resultados}

\section{Evaluación de los componentes de resiliencia comunitaria ante
desastres}\label{evaluaciuxf3n-de-los-componentes-de-resiliencia-comunitaria-ante-desastres}

Los resultados obtenidos mediante la calificación de la comunidad dan
cuenta de la multitemporalidad. Se identificó el nivel de resiliencia
comunitaria ante desastres en el año 2011, en el cual ocurrió el evento,
y en el año 2021, con el fin de hacer un contraste de las condiciones y
poder generar recomendaciones. En la \textbf{?@tbl-3} se presentan los
resultados por componentes y por temporalidad.

\textbf{Tabla 3.} Calificaciones de las áreas temáticas y los
componentes del ARC-D {[}6{]}. Fuente: Autores.

\begin{longtable}[]{@{}
  >{\raggedright\arraybackslash}p{(\linewidth - 6\tabcolsep) * \real{0.2500}}
  >{\raggedright\arraybackslash}p{(\linewidth - 6\tabcolsep) * \real{0.2500}}
  >{\centering\arraybackslash}p{(\linewidth - 6\tabcolsep) * \real{0.2500}}
  >{\centering\arraybackslash}p{(\linewidth - 6\tabcolsep) * \real{0.2500}}@{}}
\toprule\noalign{}
\begin{minipage}[b]{\linewidth}\raggedright
Calificaciones de los componentes de resiliencia comunitaria ante
desastres.
\end{minipage} & \begin{minipage}[b]{\linewidth}\raggedright
\end{minipage} & \begin{minipage}[b]{\linewidth}\centering
\end{minipage} & \begin{minipage}[b]{\linewidth}\centering
\end{minipage} \\
\midrule\noalign{}
\endhead
\bottomrule\noalign{}
\endlastfoot
\textbf{Áreas Temáticas} & \textbf{Componentes de Resiliencia a
Desastres} & \textbf{Calificación Resiliencia: (Evaluación antes de la
explosión - 2011)} & \textbf{Calificación Resiliencia: (Evaluación
después de la explosión - 2021)} \\
1. Comprender el riesgo de desastres & 1. Evaluación comunitaria
participativa de riesgo & 2 & 4 \\
& 2. Evaluación científica del riesgo & 2 & 3 \\
& 3. Diseminación de información en RRD & 2 & 4 \\
& 4. Educación de los niños en RRD & 1 & 3 \\
2. Fortalecer la Gobernanza para Gestionar el Riesgo de Desastres & 5.
RRD en la planificación del desarrollo & 1 & 2 \\
& 6. RRD en la planificación territorial & 1 & 1 \\
& 7. Toma comunitaria de decisiones & 1 & 4 \\
& 8. Inclusión de grupos vulnerables & 2 & 3 \\
& 9. Participación de las mujeres & 2 & 5 \\
& 10. Conocimiento de derechos e incidencia & 2 & 4 \\
& 11. Alianzas para la RRD y recuperación & 1 & 4 \\
3.Reducir la Vulnerabilidad a Desastres para Mejorar la Resiliencia &
12. Gestión ambiental sostenible & 2 & 3 \\
& 13. Seguridad y gestión del agua & 2 & 4 \\
& 14. Acceso y conciencia de la salud & 2 & 4 \\
& 15. Suministro seguro de alimentos & 2 & 4 \\
& 16. Prácticas de medios de vida resistentes a amenazas & 1 & 2 \\
& 17. Acceso a mercado & 1 & 3 \\
& 18. Acceso a servicios financieros & 1 & 2 \\
& 19. Protección de ingresos y activos & 2 & 2 \\
& 20. Acceso a protección social & 2 & 3 \\
& 21. Cohesión social y prevención de conflictos & 4 & 2 \\
& 22. Infraestructura Crítica & 1 & 3 \\
& 23. Vivienda & 1 & 3 \\
4. Mejorar la Preparación ante Desastres para Respuestas Efectivas y
para ``ReconstruirMejor''\,'' después de la Recuperación & 24.
Planificación de contingencia y recuperación & 1 & 3 \\
& 25. Sistema de alerta temprana & 1 & 4 \\
& 26. Capacidad de preparación, respuesta y recuperación temprana & 2 &
4 \\
& 27. Servicios de salud durante emergencias & 2 & 4 \\
& 28. Servicios de educación en emergencias & 2 & 4 \\
& 29. Infraestructura en emergencias & 1 & 1 \\
& 30. Liderazgo y voluntariado en respuesta y recuperación & 3 & 4 \\
TOTAL & & 50 & 96 \\
\textbf{NIVEL DE RESILIENCIA COMUNITARIA} & & \textbf{30.08\% BAJA} &
\textbf{60,18\% MEDIANA} \\
\end{longtable}

\textbf{4.1}. \textbf{Evaluación de resiliencia comunitaria ante
desastres Año 2011 en la Comuna 10}

En los resultados de medición de resiliencia comunitaria se observa que
para el año 2011, se obtuvo una calificación del 30,8\% categorizada
como mínima resiliencia según los criterios de la Table~\ref{tbl-1}. La
comunidad calificó 14 componentes con un nivel de 1, valoración que
corresponde a la mínima resiliencia, que se describe como poca
conciencia del problema o poca motivación para abordarlo, lo que limita
la respuesta durante la crisis.\\
Lo anterior concuerda con lo expresado en el grupo focal, por parte de
los líderes, quienes aseguran que, en ese año, no se contaba con un
escenario fortalecido de gobernanza para gestionar el riesgo de
desastres.\\
Asimismo, el área temática de la reducción de la vulnerabilidad
evidenció la falta de preparación del municipio y su comunidad para
abordar el evento de riesgo tecnológico que cobró 33 vidas en la Comuna
10. Frente a la preparación del desastre para una respuesta efectiva, se
considera una mínima planificación de contingencia y recuperación
durante el año 2011 en la comuna, lo que conllevó a agudizar el evento
(\textbf{?@fig-7}).\\
Por su parte, la medición arrojó 15 componentes con un nivel 2, los
cuales se describen como de baja resiliencia, en donde se tiene
conciencia del problema, se cuenta con capacidad para actuar, pero de
manera limitada, con intervenciones fragmentadas y a corto plazo. Esto,
en relación con lo manifestado por la comunidad, en temas de comprensión
del riesgo, específicamente, no se contaba con estudios detallados de la
comunidad sobre riesgos de desastres. Por su parte, se evidenció que el
tema de inclusión de grupos vulnerables y mujeres, así como conocimiento
de derechos e incidencia, se ubicaba en esta calificación, a pesar de
que existía una consolidada organización comunitaria
(\textbf{?@fig-7}).\\
El tema del liderazgo y voluntariado, arrojó una puntuación con un valor
de 3, que corresponde a mediana resiliencia, en dónde se caracteriza por
el desarrollo e implementación de soluciones, así como la capacidad de
actuar, siendo las intervenciones numerosas y de largo plazo. Esta
calificación la otorgó la comunidad en respuesta a la recuperación,
debido a las redes de apoyo de la comunidad que hicieron más fácil la
respuesta y la recuperación del desastre en el año 2011.\\
Finalmente se obtuvo una calificación con un nivel de 4, en donde la
comunidad le otorgó la categoría de resiliencia al componente dirigido a
la cohesión social y prevención de conflictos, dado a que en la Comuna
10 siempre ha existido una organización social y comunitaria,
fortalecida y participativa lo que en el 2011 les permitió hacerle
frente al desastre exigir a la institucionalidad una respuesta oportuna,
y generar redes de apoyo para superar la crisis. Para la comunidad, este
componente es el más importante y al único que le dieron una alta
puntuación reconociendo el valor que tuvieron muchos líderes. La
categoría de resiliencia hace referencia a la coherencia e integración,
intervenciones amplias, cubriendo los mayores aspectos del problema y
ligadas a una estrategia coherente y de largo plazo.\\
\pandocbounded{\includegraphics[keepaspectratio]{index_files/mediabag/Yf0-VODZazEAAAAASUVO.png}}\\
\textbf{Figura 7.} Resultado de la evaluación de resiliencia comunitaria
ante desastres del año 2011. (Fuente: autores, procesamiento de datos
CommCare -- GOAL).

\textbf{4.2}. \textbf{Evaluación de resiliencia comunitaria ante
desastres Año 2011 en la Comuna 10}

Para el año 2021 la comunidad calificó los componentes de acuerdo con la
experiencia vivida, y a los aprendizajes que han surgido después del
desastre durante esta década, la calificación más baja con un nivel de
1, categoría de mínima resiliencia, lo otorgaron al componente de
planificación del desarrollo, en donde sigue siendo muy corta la
intervención de la gestión del riesgo de desastres como un instrumento
para mejorar las condiciones sociales de la comunidad, así como la
destinación de recursos y acciones por parte de las juntas de acción
comunal.\\
De esta manera, también le otorgaron un valor de una mínima resiliencia
a la infraestructura en emergencia, debido a que, a pesar del evento del
2011, aún no se cuenta con viviendas sismorresistentes, infraestructura
crítica pensada para resistir a un evento de tipo tecnológico, y se
encuentra con muchos factores de riesgo alrededor de la comuna. Esta
categoría evidencia poca conciencia del problema y las acciones
limitadas a respuestas durante crisis.\\
La comunidad otorgó un nivel de 2 en la categoría de baja resiliencia a
cinco componentes para el año 2021, enfocados en la planificación del
desarrollo y la gestión del riesgo de desastres, así como a las
prácticas de medios de vida resistente a amenazas, la protección de
ingresos y activos y recursos financieros para enfrentar una crisis y la
cohesión social y prevención de conflictos. Lo anterior se debe, a que
las comunidades ven muy importante que la planificación del desarrollo
de los barrios de la comuna afecte positivamente la gestión del riesgo
de desastres, debido a los factores de riesgo que se encuentran en la
zona , a su vez el tema de acceso a servicios financieros y protección
de ingresos y activos, es algo que no está resuelto, debido a que no se
ha pensado estrictamente para sus comunidades, este tema frente a la
cohesión social y prevención de conflictos, se manifestó una gran
preocupación debido a que a medida que pasa el tiempo los lazos
comunitarios son más difíciles de tejer y existe una problemática
estructural frente a los temas de seguridad paz y convivencia.\\
La comunidad otorgó a nueve componentes de la evaluación, un nivel de 3,
que corresponde a la categoría de mediana resiliencia, los cuales se
relacionan a componentes como: la comprensión del riesgo de desastres en
la reducción de la vulnerabilidad para mejorar la resiliencia, en dónde
resaltan como prioridad, educar a los niños en la gestión del riesgo de
desastres, así como generar una evaluación científica del riesgo en sus
comunidades en acompañamiento con la institucionalidad. Por otro lado,
la participación de los grupos vulnerables en la gestión del riesgo está
abierta y existen los espacios; no obstante, es poco atractiva para
ciertos grupos poblacionales. De manera seguida, se evidencia frente a
la gestión ambiental sostenible que las acciones que se generan en torno
al cuidado de los cuerpos de agua las realiza la asociación de
acueductos comunitarios, siendo los únicos actores que realizan campañas
frente a este tema. Manifiestan importante tener acceso a mercado y a
protección social en el momento que vuelva a ocurrir un evento, ya que
esto es uno de los temas más frágiles y necesarios para la comunidad y
finalmente se encuentra en esta escala los componentes de
infraestructura crítica y vivienda, siendo importante el reforzamiento
ante cualquier factor amenazante en la comuna.\\
La comunidad le otorgó a trece componentes de la evaluación, un nivel de
4, considerándolos en la categoría de resilientes, esta calificación fue
dirigida a componentes como: evaluación comunitaria participativa del
riesgo, la diseminación de la información de la gestión del riesgo de
desastres hacia la comunidad, la toma comunitaria de decisiones de
manera abierta y participativa, y el conocimiento de derechos e
incidencia, alianzas para la reducción del riesgo y la recuperación, el
acceso a servicios básicos en un evento y acciones enfocadas a la
planificación de la contingencia, sistema de alerta temprana, capacidad
de preparación respuesta y recuperación temprana, liderazgo en temas de
voluntariado en respuesta y recuperación. Esta calificación permitió
contrastar ciertos componentes del año 2011, siendo reconocido por la
comunidad, la mejora en ciertos procesos de la gestión de riesgo de
desastres, lo que ha permitido que la Comuna 10 cuente con mejores
condiciones de resiliencia, cabe resaltar que el evento a pesar de que
fue un momento difícil y se perdieron vidas de amigos, familiares y
vecinos, ha permitido a las comunidades resurgir del proceso negativo y
generaron los tejido comunitario enfocado a la reducción de riesgo de
desastres.\\
La comunidad otorgó un valor de 5, de alta resiliencia, a la
participación de la mujer en la gestión del riesgo de desastres, ya que
en el año 2021 el porcentaje de mujeres que participan en la toma de
decisiones aumentó significativamente.

\pandocbounded{\includegraphics[keepaspectratio]{index_files/mediabag/wNnGGhuWgjVLwAAAABJR.png}}\\
\textbf{Figura 8.} Resultado de la evaluación resiliencia comunitaria
ante desastres año 2021. (Fuente: autores, procesamiento de datos
CommCare -- GOAL).

\bookmarksetup{startatroot}

\chapter{5. Discusión}\label{discusiuxf3n}

En la \textbf{?@fig-9} se evidencian los contrastes entre las dos
evaluaciones, hallando contrastes significativos en ciertos componentes
como la diseminación de la información de reducción del riesgo de
desastres, el cual pasó de un nivel 1 a 4, debido al proceso que ha
surgido con la comunidad, sobre capacitación, formación y
sensibilización en materia de reducción del riesgo de desastres. Esto ha
permitido que hoy en día se cuente con procesos sociales, organizaciones
y veedurías ciudadanas en la gestión del riesgo de desastres.

Lo anterior está muy relacionado con la toma de decisiones comunitaria,
esto se evidencia, del paso de 1 a 4, en dónde la comunidad ha tenido un
nivel de empoderamiento en las decisiones que se tomen en su territorio,
participando en las instancias dónde se toman decisiones y se puede
incidir para mejorar las condiciones de vida de las personas.

El caso atípico, fue el valor dado a la reducción del riesgo de
desastres en la planificación territorial, debido a que los líderes
consideran que, en estos diez años, no se ha tenido un avance
significativo en este componente, evidenciado un retroceso, dónde las
Juntas de Acción Comunal, no consideran este tema relevante en algunos
territorios que articulen la gestión de riesgo con la ejecución
presupuestal ni en sus planes de acción.

La calificación más alta otorgada, fue la participación de las mujeres,
pasando de un nivel de 2 a 5, siendo significativo su avance, ya que son
las mujeres las que en la actualidad más participan en los temas de
gestión de riesgo de desastres en la comuna 10 y están liderando
diferentes procesos.

En el caso de las alianzas para la reducción del riesgo de desastres y
recuperación, se tuvo un paso de 0 a 4 en la calificación, lo que
muestra la mejora de las condiciones y capacidades de recuperación en la
comuna, no solo de las comunidades sino también de la institución.
Frente a la cohesión social y prevención de conflictos, se consideró que
en el 2011 las condiciones de seguridad eran mejores que en la
actualidad, pasando de un nivel de 4 a 2.

Finalmente, otra de las grandes diferencias encontradas, se evidencia en
el sistema de capacidad de preparación, respuesta y recuperación
temprana, pasó de una calificación de 2 a 4, ya que dichas capacidades,
han mejorado notablemente en la comuna y en el municipio, viéndose
reflejado en la Estrategia Municipal de Respuesta a Emergencias (EMRE) y
en la organización comunitaria.

\pandocbounded{\includegraphics[keepaspectratio]{index_files/mediabag/FsnAAAAAElFTkSuQmCC.png}}

\textbf{Figura 9.} Análisis comparativo evaluación de la resiliencia
comunitaria ante desastres año 2011 y 2021.

\bookmarksetup{startatroot}

\chapter{6. Conclusiones}\label{conclusiones}

El estudio tuvo como objetivo general evaluar la resiliencia comunitaria
frente a desastres de tipo tecnológico en la Comuna 10 de Dosquebradas,
mediante la aplicación de la herramienta ARC-D en dos momentos
temporales, 2011 y 2021. Esta comparación multitemporal permitió
identificar avances y limitaciones en las capacidades comunitarias,
evidenciando una transición de baja a mediana resiliencia en el periodo
de análisis frente a estresores de tipo tecnológico.\\
La aplicación de la ARC-D demostró ser una herramienta de fácil
comprensión para la comunidad y facilitó espacios participativos de
reflexión sobre la gestión del riesgo. A diferencia de metodologías como
la Evaluación de Vulnerabilidades y Capacidades (AVC) de la Cruz Roja
Internacional, que busca sensibilizar y planificar actividades
orientadas a reducir vulnerabilidades, la ARC-D se centra en medir
componentes de resiliencia comunitaria de forma estructurada y
participativa, con un enfoque en la percepción del riesgo de la
comunidad respecto al riesgo tecnológico presente en su entorno,
permitiendo además su uso en análisis multitemporales, en este caso
específico motivo de los antecedentes presentados por la Comuna 10 en el
evento por explosión en el año 2011. La diferencia metodológica aportada
en el presente análisis, representa un aporte significativo al campo de
la percepción del riesgo tecnológico y complementa los enfoques
tradicionales. Adicionalmente, este enfoque multitemporal de la medición
de resiliencia no se había presentado en aplicaciones anteriores del
ARC-D.\\
Los resultados de la medición de resiliencia comunitaria mostraron
estabilidad en tres componentes a lo largo del tiempo: planificación
territorial, protección de ingresos e infraestructura de emergencias. La
permanencia del componente refleja tanto limitaciones estructurales,
como la ausencia de cambios sustantivos en la planificación del uso de
suelo, la persistencia de altos niveles de informalidad laboral y la
falta de infraestructura comunitaria para la atención de emergencias. Lo
anterior indica que, pese a los avances en otros ámbitos, existen
factores estructurales que requieren políticas sostenidas y articuladas
a nivel municipal.\\
Un hallazgo relevante fue el retroceso en el componente de cohesión
social y prevención de conflictos, que evidencia un deterioro en la
percepción de seguridad y en las medidas comunitarias de resolución de
conflictos. La situación social refleja tensiones sociales que deben ser
consideradas en las estrategias de gestión del riesgo, pues la cohesión
social es un factor determinante de la resiliencia, tal como lo
reconocen marcos internacionales como el Marco de Sendai.\\
En contraste, se observaron mejoras notorias en la participación de las
mujeres y en la gobernanza comunitaria, alcanzando incluso niveles altos
de resiliencia, este avance destaca el rol protagónico de las mujeres en
los procesos comunitarios y la importancia de fortalecer su inclusión en
la toma de decisiones.\\
De fondo respecto a la resiliencia comunitaria, se evidenció un cambio
general en los niveles de resiliencia, pasando de baja en 2011 a mediana
en 2021, lo cual muestra una evolución positiva en la capacidad de la
comunidad para enfrentar riesgos tecnológicos.\\
En síntesis, la evaluación multitemporal de la resiliencia comunitaria
mediante la herramienta ARC-D permitió identificar tanto avances como
limitaciones en la Comuna 10 de Dosquebradas. Estos resultados
contribuyen al conocimiento aplicado sobre resiliencia comunitaria
frente a riesgos tecnológicos, un campo aún poco explorado en Colombia,
y ofrecen insumos prácticos para el diseño de políticas públicas y
estrategias comunitarias. Al articularse con las prioridades del Marco
de Sendai y las Políticas Nacionales de Gestión del Riesgo de Desastres
en Colombia, se resalta la necesidad de fortalecer la gobernanza,
reducir vulnerabilidades estructurales y promover la cohesión social
como ejes centrales para mejorar la resiliencia de las comunidades
urbanas expuestas a riesgos tecnológicos.

\begin{tcolorbox}[enhanced jigsaw, arc=.35mm, colback=white, opacityback=0, colframe=quarto-callout-important-color-frame, left=2mm, toprule=.15mm, rightrule=.15mm, breakable, leftrule=.75mm, bottomrule=.15mm]

\vspace{-3mm}\textbf{Puntos Clave}\vspace{3mm}

\begin{itemize}
\tightlist
\item
  En 2011 la resiliencia comunitaria frente a riesgo tecnológico en la
  Comuna 10 era baja, lo que agravó los impactos del desastre presentado
  en la explosión
\item
  Entre 2011 y 2021 la resiliencia aumentó a un nivel medio gracias a
  los procesos de recuperación postdesastre, los aprendizajes colectivos
  y el mayor liderazgo de las mujeres
\item
  Para la gestión del riesgo actual, persisten limitaciones en
  planificación territorial, protección de ingresos e infraestructura de
  emergencias, sin embargo, pueden existir fortalezas debido a la
  obtención de capacidades que ha obtenido la comunidad después de la
  explosión
\end{itemize}

\end{tcolorbox}

\begin{tcolorbox}[enhanced jigsaw, colback=white, opacityback=0, bottomtitle=1mm, toprule=.15mm, colbacktitle=quarto-callout-tip-color!10!white, titlerule=0mm, toptitle=1mm, breakable, leftrule=.75mm, coltitle=black, arc=.35mm, opacitybacktitle=0.6, colframe=quarto-callout-tip-color-frame, left=2mm, title=\textcolor{quarto-callout-tip-color}{\faLightbulb}\hspace{0.5em}{Recomendaciones para Tomar Decisiones}, bottomrule=.15mm, rightrule=.15mm]

\begin{itemize}
\tightlist
\item
  Replicar la aplicación de la herramienta ARC-D en otros municipios
  para evaluar y fortalecer la resiliencia comunitaria frente a riesgos
  tecnológicos
\item
  Incluir el riesgo tecnológico de forma explícita en los planes de
  gestión del riesgo y en los instrumentos de planificación territorial
\item
  Socializar y sensibilizar a la comunidad sobre el riesgo tecnológico,
  vinculándolo a la vida cotidiana, los medios de vida y la movilidad
\item
  Usar la ARC-D como insumo para diagnósticos locales en el marco de la
  iniciativa MCR2030, articulando sus resultados con los planes
  municipales
\item
  En el caso de Dosquebradas, expandir el ejercicio a otros escenarios
  de riesgo para aprovechar aprendizajes y fortalecer la preparación
  comunitaria
\end{itemize}

\end{tcolorbox}

\begin{center}\rule{0.5\linewidth}{0.5pt}\end{center}

\section{Conflicto de Intereses}\label{conflicto-de-intereses}

Los autores no declaran conflicto de intereses.

\section{Uso de Herramientas de Inteligencia
Artificial}\label{uso-de-herramientas-de-inteligencia-artificial}

Durante la preparación de este capítulo, los autores utilizaron ChatGPT
versión XX (https://chatgpt.com) con el fin de recibir sugerencias de
redacción y corrección del estilo científico.

Los revisores expresan que no utilizaron herramientas de IA en el
proceso de evaluación del manuscrito.

\section{Contribución de Autoría
CREDIT}\label{contribuciuxf3n-de-autoruxeda-credit}

Todos los autores contribuyeron a la conceptualización y desarrollo de
este proyecto.

\section{Agradecimientos}\label{agradecimientos}

Queremos agradecer a los líderes de la Comuna 10 del municipio de
Dosquebradas, por su apoyo y participación en las reuniones previas y
grupo focal, lo cual fue de vital importancia para la obtención de los
resultados. Especialmente, queremos agradecer a Don Elmer Castañeda y
Diego Buitrago, líderes y expertos comunitarios en la gestión del riesgo
de desastres en la Comuna 10, los cuales apoyaron de manera
desinteresada esta evaluación de resiliencia comunitaria, viéndola como
una oportunidad para el proceso de la gestión de riesgo.

\section{Identificación de Autores}\label{identificaciuxf3n-de-autores}

\textbf{Evelin Langebeck Cuéllar}
\url{https://scholar.google.es/citations?user=FzyY5g8AAAAJ&hl=es}

\textbf{Nicolás Giraldo Hernández}
\url{https://scienti.minciencias.gov.co/cvlac/visualizador/generarCurriculoCv.do?cod_rh=0001984270}

\begin{center}\rule{0.5\linewidth}{0.5pt}\end{center}

\bookmarksetup{startatroot}

\chapter{Referencias}\label{referencias}

\begin{enumerate}
\def\labelenumi{\arabic{enumi}.}
\item
  Cruz Roja Internacional. (2006). \emph{¿Cómo se hace un AVC? Guía
  práctica para el personal y los voluntarios de la Cruz Roja y de la
  Media Luna Roja}. Cruz Roja.
\item
  Ciccotti, L., Rodrigues, A., Boscov, M., \& Gunther, W. (2020).
  Building indicators of community resilience to disasters in Brazil: A
  participatory approach. \emph{Ambiente \& Sociedade, 23}, 1-24.
  \url{https://doi.org/10.1590/1809-4422asoc20190151vu2020L3AO}
\item
  Klimsa, C. (2019). \emph{Avances en los lineamientos metodológicos
  para aproximarse a la medición de resiliencia}. CEPAL.
  \url{https://repositorio.cepal.org/handle/11362/45042}
\item
  ONU (Oficina de las Naciones Unidas para la Reducción del Riesgo de
  Desastres). (2009). \emph{Terminología sobre reducción del riesgo de
  desastres}. UNISDR.
  \url{https://www.unisdr.org/files/7817_UNISDRTerminologySpanish.pdf}
\item
  ONU (Oficina de las Naciones Unidas para la Reducción del Riesgo de
  Desastres). (2016). \emph{Informe del grupo de trabajo
  intergubernamental de expertos de composición abierta sobre los
  indicadores y la terminología relacionados con la reducción del riesgo
  de desastres}. Naciones Unidas.
\item
  McCaul, B., \& Mitsidou, A. (2016). \emph{Análisis de la resiliencia
  de las comunidades ante desastres}. GOAL.
\item
  GOAL. (2015). \emph{Herramienta para medir la resiliencia comunitaria
  ante desastres}. GOAL.
\item
  GOAL. (2016). \emph{Análisis de la resiliencia de las comunidades ante
  los desastres: Caja de herramientas ARC-D y manual de guía al
  usuario}. GOAL.
\item
  Alcaldía de Dosquebradas, Dirección de Gestión del Riesgo -- DIGER.
  (2020). \emph{Plan municipal de gestión del riesgo de desastres}.
  Alcaldía de Dosquebradas.
\item
  Ecopetrol S.A., \& Corporiesgos. (2012). \emph{Formación de grupos
  comunitarios, líderes gestores de gestión de riesgos: Análisis de
  riesgos en el territorio. Productos comunitarios}. Ecopetrol.
\item
  Vásquez, J. (2020). \emph{Análisis de la evaluación al plan en la
  gestión del riesgo de desastres antes y después del 2011 en la Comuna
  10 del municipio de Dosquebradas} {[}Trabajo de investigación{]}.
\item
  Unidad Nacional de Gestión del Riesgo de Desastres. (2018). \emph{Lo
  que usted debe saber sobre riesgo tecnológico}. UNGRD.
\item
  Ecopetrol S.A., Veeduría Ciudadana, \& Fundación Social Cooplarosa.
  (2015). \emph{Huellas de esperanza}. Ecopetrol.
\item
  Sandoval-Díaz, J., Navarrete Muñoz, M., \& Cuadra Martínez, D. (2023).
  Revisión sistemática sobre la capacidad de adaptación y resiliencia
  comunitaria ante desastres socionaturales en América Latina y el
  Caribe. \emph{Revista de Estudios Latinoamericanos sobre Reducción del
  Riesgo de Desastres, 7}(2), 187-203.
  \url{https://doi.org/10.55467/reder.v7i2.130}
\item
  Pell del Río, S. M., Valdés Santiago, D., Gil Rodríguez, A. L., Amador
  Romero, F. J., Cairo Pell, K. S., Paneque Quevedo, A. A., Lorenzo
  Ruíz, A., \& Febles Elejalde, M. M. (2021). Risk perception during
  confinement by COVID-19 in a Cuban sample: Preliminary results.
  \emph{Anales de la Academia de Ciencias de Cuba, 11}(1).
  \url{http://www.revistaccuba.sld.cu/index.php/revacc/article/view/895}
\item
  Frómeta-Alfaro, M., \& Guardado-Lacaba, R. (2017). Percepción del
  riesgo: Su rol ante el cambio climático, sus efectos y la adaptación.
  \emph{Revista de Innovación Social y Desarrollo, 2}(1), 96-108.
  \url{https://revistas.uh.cu/risd/article/view/3316}
\item
  Jaque Castillo, E. D. (2021). \emph{Resiliencia frente a riesgos de
  desastres en la ciudad de Linares, Chile: Evaluación a través del
  modelo de las Naciones Unidas} {[}Tesis de Maestría, Universidad de
  Concepción{]}. \url{https://repositorio.udec.cl/handle/11594/9470}
\item
  Yánez, M., Martelo, J., \& Rodríguez, H. (2020). Cálculo y análisis de
  la resiliencia. \emph{Sociedad y Economía, 41}, 64-87.
  \url{https://doi.org/10.25100/sye.v0i41.8823}
\end{enumerate}

\bookmarksetup{startatroot}

\chapter*{References}\label{references}
\addcontentsline{toc}{chapter}{References}

\markboth{References}{References}

\phantomsection\label{refs}


\backmatter


\end{document}
